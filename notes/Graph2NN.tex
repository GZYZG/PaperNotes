\begin{center}

  \begin{tabular}{rp{6cm}lp{12cm}}%{rl}

  % after \\: \hline or \cline{col1-col2} \cline{col3-col4} ...

  论文地址:& \href{https://arxiv.org/abs/2007.06559}{https://arxiv.org/abs/2007.06559} \\

  源码:& \href{https://github.com/facebookresearch/graph2nn}{graph2nn} \\

  slides:& \href{https://cs.stanford.edu/~jiaxuan/files/Graph_Structure_of_Neural_Networks_slides.pdf}{{\footnotesize Graph Sturcture of Neural Networks}}\\

  关键词:& \textbf{Neural Networks, Graph Analysis} \\

  写于:& \date{2020-10-14}

  \end{tabular}

\end{center}


这篇论文\cite{you2020graph}从图的角度来解释神经网络(Neural Networks, NN),用relational graph来对NN进行建模,并在relational graph的基础上定义了NN训练的过程。在最短路径和聚集系数的基础上分析NN的性能。以这样的视角来解释和分析NN是一个新颖的角度,\tred{为设计NN和解释NN提供了新的思路}。

\paragraph{如何建模NN为图?}引用文中的一句话:focus on message exchange, rather than just on directed data flow。建模的关键出发点是对NN中神经单元之间信息的交换,而不是信息的流向。想象如果给你一个图$\mathcal{G(V, E)}$,怎么把它看作一个NN呢,NN的训练如何在其上展开呢?论文中把$\mathcal{G}$称为relational graph。一个NN会有多层,可是一个图只有这么多结点,那么怎么训练这个“图化”后的NN呢?论文中使用round来对应NN中的层。则NN中信息在层间传递时,在$\mathcal{G}$中看起来是这样的:
$$
\boldsymbol{x}_v^{(r+1)} = AGG^{(r)} ( { f_v^{(r)} ( \boldsymbol{x}_u^{(r)}, \forall u \in N(v) ) } )
$$
其中,$\boldsymbol{x}_v^{(r)}$是结点$v$在第r round时的特征,$N(v)$是结点$v$的邻居(relational  graph中的结点都是自环的)。$AGG$类似于NN中的激活函数,$f$则类似于NN中的对上一层的输入进行线性组合或这卷积操作等。这看起来和基于消息传递的GNN模型看起来是很相似的。
\newline

论文中基于relational graph对现有的一些NN进行了分析,如固定宽度的MLP、可变宽度的MLP、ResNet的一些变体等,使用relational graph进行训练,并分析了NN“图化”后的结构与性能之间的关系。

\paragraph{方法解决的问题/优势}
\begin{itemize}
	\item 开创性的将NN视作图数据进行分析(哎,我也想到了这个点子)
	\item 以图的视角分析NN或许能够对NN的设计和理解产生很大影响
%	\item 

\end{itemize}

\paragraph{方法的局限性/未来方向}
\begin{itemize}
	\item 个人认为论文中对NN建模为relational graph的想法很不错,也很符合现在的GNN做法,但是对于复杂的NN,以及NN中的一些特殊的操作,论文中的relational graph就有点简单了,\tbc{red}{难以表现复杂的NN结构}
	\item 对relational graph的分析还比较少,局限于最短距离和聚集系数,可以分析图中的局部性的结构,如团、子图、motif等,以及一些其他的角度
	\item \tbc{red}{有期望能够打通GNN和NN之间的联系,或许在此之上能够提出新的深度学习模型!!!}
	\item 在论文基础上对NN进行解释
\end{itemize}



