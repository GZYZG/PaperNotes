\begin{center}

  \begin{tabular}{rp{16cm}lp{20cm}}%{rl}

  % after \\: \hline or \cline{col1-col2} \cline{col3-col4} ...

  论文地址:& \href{https://arxiv.org/pdf/1801.10247.pdf}{https://arxiv.org/pdf/1801.10247.pdf} \\
  来源:& ICLR, 2018 \\
  作者:& Jie Chen, Tengfei Ma, Cao Xiao \\

  源码:& \href{https://github.com/matenure/FastGCN}{FastGCN} \\

%  slides:& \href{http://yunshengb.com/wp-content/uploads/2017/03/nips_2018_r2l_workshop_talk.pdf}{{\footnotesize Convolutional Set Matching for Graph Similarity}}\\

  关键词:& \textbf{GCN, GNN} \\

  写于:& \date{2021-01-21}

  \end{tabular}

\end{center}

该论文\cite{chen2018fastgcn}针对GCN训练和学习过程中地计算量大和耗时提出了integral transform的方法。本文所针对的Graph主要是稠密图和符合幂律的Graph,当对其中一个结点卷积时,所覆盖的结点集可能已经很大了。同时,FastGCN是inductive的。

本文的重点是对结点的采样以及从积分的角度来看待卷积层。

这篇论文没看太懂,可见博客:\href{https://zhuanlan.zhihu.com/p/106226258}{源码分析-FastGCN}、\href{https://blog.csdn.net/yyl424525/article/details/101101079}{FastGCN}。


