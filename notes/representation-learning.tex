\begin{center}

  \begin{tabular}{rp{16cm}lp{20cm}}%{rl}

  % after \\: \hline or \cline{col1-col2} \cline{col3-col4} ...

  论文地址:& \href{https://ieeexplore.ieee.org/stamp/stamp.jsp?tp=&arnumber=6472238}{https://ieeexplore.ieee.org/stamp/stamp.jsp?tp=\&arnumber=6472238} \\
  来源:& IEEE TRANSACTIONS ON PATTERN ANALYSIS AND MACHINE INTELLIGENCE, 2013 \\
  作者:& Yoshua Bengio, Aaron Courville, and Pascal Vincent \\

 % 源码:& \href{xxx}{xxx} \\

%  slides:& \href{http://yunshengb.com/wp-content/uploads/2017/03/nips_2018_r2l_workshop_talk.pdf}{{\footnotesize Convolutional Set Matching for Graph Similarity}}\\

  关键词:& \textbf{deep learning, representation learning, feature learning} \\

  写于:& \date{2021-01-26}

  \end{tabular}

\end{center}

该论文\cite{6472238}写于2013年,虽然有一点久远了,但全文内容十分充实、全面,不愧是Bengio巨佬的论文。该文主要对无监督的表征学习进行了总结,内容涵盖了概率模型、自编码器、流形学习以及神经网络。

机器学习的效果严重依赖于输入的数据 --- 即特征工程,在机器学习相关的任务中大部分的时间其实都花在了处理数据、构建特征上。特征工程通常作为下游任务的输入,对于构建的特征,有一个很通用的标准:要有区分性。

\paragraph{表征学习的应用}
\subparagraph{语音识别与信号处理}
\subparagraph{目标识别}
\subparagraph{自然语言处理}
\subparagraph{多任务学习、迁移学习、领域适应}

\paragraph{表征学习中的Priors}
\subparagraph{Smothness}
输入空间中相近的数据点在表征空间中也应该是相近的。相近的概念要看各个空间中距离是如何定义的。这是最基本的一个prior,但可能会导致curse of dimensionality。通常输入空间中样本的分布是很复杂的,在表征空间中会对输入空间进行一定的压缩,这会导致在表征空间中产生一些wrinkles --- 可能有一些导数很大或者不可导的区域。为了在表征空间中尽量连续,则需要大量的数据来“抹平”这些wrinkles。当输入空间的维数增大时,为了“抹平”wrinkles所需的数据将呈指数形式增长 --- curse of dimensionality。

\subparagraph{Multiple explanatory factors}
输入的数据可能是由一些潜在的因素生成的 --- “有一双魔掌在控制着世间万物的生灭”。

\subparagraph{A hierarchical organization of explanatory factors}
潜在因素可以通过层次化的结构进行组织,即我们对数据的抽象是层次化的、逐步的。

\subparagraph{Semi-supervised learning}
当我们使用数据去预测目标时,控制输入数据的潜在因素同样也能控制目标。对$P(X)$有用的表征同样对$P(Y|X)$也是有用的。其实,我感觉这也可以从层次化的角度来看,目标可以看作输入的更深层次的抽象,输入和目标都是受潜在因素控制的,只不过二者的抽象程度不同。

\subparagraph{Shared factors across tasks}
不同的任务之间是会共享一些潜在因素的,这也是迁移学习有效果的原因之一。

\subparagraph{Manifold}
数据空间中,样本点并不是随机或者均匀分布的,而是呈现聚集的状态,数据会集中地占据空间的一部分。不同类型的数据之间数据点密度是较低的。

\subparagraph{Natural clustering}
同类的数据在输入/表征空间中应该是自然聚集的。

\subparagraph{Temporal and spatial coherence}
输入空间中的微小变化也对应着目标空间中的微小变化,与函数的连续性很相似。

\subparagraph{Sparsity}
给定一个样本点$x$,只有一部分因素与之相关,具体来看,表征向量大部分维上取值维0,既可以表示为表征向量对输入的微小变化是不敏感的。

\subparagraph{Simplicity of factor dependencies}
良好的表征,因素之间的关系应该是尽量简单的,如相互独立、线性关系等。

\paragraph{What makes a Representation Good?}
首先,当然应该包含上述的Priors。其次,还有以下因素:
\subparagraph{Distributed representation}
\subparagraph{Depth and Abstraction}
通常表征学习模型具有一定的深度 --- 在学习数据表征的过程中是有一定层次的,能够帮助产生逐步抽象的数据。抽象的数据表征对一些变化是具有不变性的,如平移、旋转、缩放等。

\subparagraph{Disentangling Factors of Variation}
潜在因素之间尽可能使独立的。

\subparagraph{Good Criteria for  Learning Representation?}

接下来论文对如何构建deep representation进行了介绍,比如通过逐层训练的方式来学习深度模型。

在表征学习中,主要可以分为两个流派:概率图模型、神经网络模型。可以简单的区分二者:隐藏单元是被视为隐变量还是可计算结点。接下来就是介绍各种模型了,其中概率模型一直没看懂,都是通过输入、表征的概率分布来建模的,一般是通过最大化似然概率来求解。

接下来是直接学习一个映射将数据从输入空间映射到表征空间,主要介绍了自编码器、regularized autoencoders。在去噪自编码器(从含噪声的样本中能够重建出无噪的样本)中,这需要借助一个假设:样本点使流形数据,数据点是聚集的。加入噪声后的数据会从一定程度上偏离流形,偏离到密度更低的区域。

其实,表征学习可以看作流形学习。有这么一个假设:真实世界中的数据所处的空间是非常大的,甚至是无限维的,数据可以用一个更低维的空间来表示。接下来论文就介绍了一些流形学习的方法。

接下来论文介绍了训练深度模型时的一些问题,如1)参数初始化的重要性;2)参数稀疏性的重要性,当大部分参数一起更新时,目标函数可能会在多个方向上移动,这使得收敛变得困难;3)数值计算的问题,如病态方程等;4)标记数据对模型的效果的提升。

这篇论文真的是非常厚重了,内容足足的,可惜还有不少内容我看不太懂,哎,巨佬就是巨佬啊,非一日之功啊!