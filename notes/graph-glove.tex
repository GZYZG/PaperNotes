\begin{center}
	\begin{tabular}{rl}
		% after \\: \hline or \cline{col1-col2} \cline{col3-col4} ...
		论文地址:& \href{https://arxiv.org/abs/2010.02598}{https://arxiv.org/abs/2010.02598} \\
		源码地址:& \href{https://github.com/yandex-research/graph-glove}{graph-glove} \\
		关键词:& \textbf{embedding, representation learning} \\
		写于:& \date{2020-10-08}
	\end{tabular}
\end{center}
传统的word embedding通常是将词嵌入到向量空间中,\cite{10.1145/219717.219748}指出words可以形成具有隐式层次结构的图,词向量的质量与选取什么样的向量空间有很大的关系。针对这个问题,该论文\cite{ryabinin2020embedding}提出了GraphGlove,将词嵌入到带权图(weighted graph)中。在word相似性及相关任务中,该论文提出的方法取得了比嵌入到向量空间中更好的效果。\\
\textbf{方法---GraphGlove}\hspace{6pt} 每个word视作带权图(weighted graph)中的一个结点,\textbf{词之间的距离使用图中结点之间的shortest path来表示(与欧氏空间中距离的定义不同)}。使用PRODIGE\cite{mazur2019vector}从数据中学习,可以得到一个带权图以及图中边的权重;再将学习得到的带权图用于GloVe\cite{pennington-etal-2014-glove}算法,学习得到在非欧空间中的词向量,关键之处是替换GloVe的损失函数中词之间距离为图中结点之间的最短距离。\\
该论文与其他embedding方法不同的之处:将word嵌入到非欧空间,学习到的带权图就是这个非欧空间。