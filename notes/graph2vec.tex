\begin{center}
  \begin{tabular}{rl}
  % after \\: \hline or \cline{col1-col2} \cline{col3-col4} ...
  论文地址:& \href{https://arxiv.org/abs/1707.05005}{https://arxiv.org/abs/1707.05005} \\
  源码地址:& \href{https://github.com/MLDroid/graph2vec_tf}{graph2vec} \\
  关键词:& \textbf{Graph Kernels, doc2vec, representation learning} \\
  写于:& \date{2020-09-30}
  \end{tabular}
\end{center}

论文\cite{DBLP:journals/corr/NarayananCVCLJ17}提出了图的embedding方法---graph2vec,该方法借鉴word2vec, doc2vec的思想,以rooted subgraph 作为图的元素,通过最大化似然概率来得到每个图的定长的embedding向量。

\par 如果熟悉word2vec的话,其实很容易就理解这篇论文的方法了(虽然graph2vec是在模仿doc2vec,但doc2vec开起来像在模仿word2vec)。因为graph2vec是在word2vec基础上做的,它们之间存在一个比较直白的对应关系,厘清它们之间的对应关系就明白graph2vec了。
\par 在介绍doc2vec和graph2vec的对应关系之前,先介绍一下rooted graph的概念。rooted subgraph是一个类似于树状的结构,可以看作一棵多叉树,树中的结点就是图中的结点,结点按照图中的邻接关系相连。可以根据图中的每个结点$v$生成一棵rooted subgraph,以$v$为根,它的子结点为它的邻居结点,每个子结点又以自己的邻居结点为子结点,如此递归定义下去就是以$v$为根的rooted subgraph。接下来介绍graph2vec和doc2vec之间的概念对应关系。
\par 在doc2vec中,每个doc(文档)被看做词序列的集合,优化文档向量的基础就是:最大化词序列和文档的共现率。graph2vec将一个graph看做一个文档,rooted graph看做文档中的词序列。所以优化graph embedding的基础就是:最大化rooted subgraph和graph的共现率。与word2vec类似,graph2vec中也使用了负采样的方法对算法进行优化。

\par 与最近使用GNN来进行graph embedding的方法相比,graph2vec是transductive的,对于未见过的graph需要重新运行一遍算法来生成embedding,与Deepwalk很像。我这里有一个想法,rooted subgraph能否通过其他形式来替代,比如node2vec中的游走策略生成的子图。
